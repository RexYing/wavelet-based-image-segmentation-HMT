\documentclass[11pt]{article}




%------------------------------------------------------------------
% PREAMBLE
%------------------------------------------------------------------

% -------------------- XeTeX vs TeX -------------------- %
% ! COMMENT THIS SECTION OUT TO ALLOW COMPILATION WITH PDFLATEX
% \usepackage[cm-default]{fontspec}
%
% \usepackage{xunicode}

% \usepackage{xltxtra}

% \defaultfontfeatures{Mapping=tex-text}

% \setromanfont{CMUSerif}

% \setsansfont{CMU Sans Serif}

% \setmonofont{CMU Typewriter Text}




% !UN-COMMENT THIS IF THE ABOVE ARE COMMENTED OUT !

% !Also add the 'pdftex' option to packages 'xcolor' and 'hyperref'!


\usepackage[utf8]{inputenc}

% smaller margin
\usepackage{fullpage}


% math
\usepackage{amsmath}
\usepackage{amsfonts}
\usepackage{amsthm}
\usepackage{mathrsfs}
\usepackage{array}
\usepackage{cancel}
\usepackage{nicefrac}
\usepackage{bigdelim}
% \usepackage{breqn}

% floats
\usepackage[section]{placeins}
\usepackage{float}
\usepackage{listings}
\usepackage[section]{algorithm}
\usepackage{algpseudocode}
\usepackage[font={small,color=darkgray},labelfont=bf,hypcap=true]{caption}
\usepackage[font={footnotesize,color=darkgray},labelfont=bf,hypcap=true]{subcaption}
% \usepackage{sidecap}

% graphics
\usepackage{graphicx}
\usepackage[hyperref,x11names]{xcolor}
% \usepackage{tikz}
% \usepackage{pgfplots}

% tables
\usepackage{multirow}
\usepackage{booktabs}

% lists
\usepackage{enumerate}
\usepackage{paralist}

% header and footer
\usepackage{fancyhdr}

% footnotes
\usepackage[bottom]{footmisc}

% ordinal numbers
\usepackage[super]{nth}

% fonts
\usepackage{textcomp}
\usepackage[resetfonts]{cmap}

% language support
% \usepackage{polyglossia}

% code
\usepackage{mcode}

% references
\usepackage[numbers,square]{natbib}
\usepackage[hyphens]{url}
\usepackage{hyperref}
\usepackage{cleveref}

\usepackage{setspace}




%------------------------------------------------------------------
% Customization
%------------------------------------------------------------------

% no indent
\setlength\parindent{0pt}

% define includegraphics search path
\graphicspath{{ figs/}{figs/pdf/}{../code/}}

% specify font formatting for listings
\lstset{basicstyle=\footnotesize\ttfamily}

% force newly defined floats to be ruled
\floatstyle{ruled}

% create a 'code' float
\newfloat{code}{tbp}{loc}[section]
\floatname{code}{Code}
\captionsetup[code]{font={small,color=darkgray},labelfont=bf,hypcap=true}

% prefix equations, figures and tables with the section number
\numberwithin{equation}{section}
\numberwithin{figure}{section}
\numberwithin{table}{section}
\numberwithin{code}{section}

% define theorems, lemmas and corollaries and make them have contiguous
% numbers
\newtheorem{theorem}{Theorem}[section]
\newtheorem{lemma}[theorem]{Lemma}
\newtheorem{corollary}[theorem]{Corollary}

% easy command for resetting counters when using section*
\newcommand{\resetcounters}{
  \setcounter{subsection}{0}
  \setcounter{subsubsection}{0}
  \setcounter{equation}{0}
  \setcounter{figure}{0}
  \setcounter{table}{0}
  \setcounter{code}{0}
}

% easy command for boldface math symbols
\newcommand{\mbs}[1]{\boldsymbol{#1}}

% define a dark gray color to be used in captions
\definecolor{darkgray}{rgb}{0.25,0.25,0.25}

% define the program float reference abbreviations
\crefname{code}{code}{codes}

% initial header and footer configuration
\setlength{\headheight}{15.2pt}
\renewcommand{\headrulewidth}{0.5pt}
\renewcommand{\footrulewidth}{0pt}
\fancyhf{}
\pagestyle{fancyplain}

% redefine 'plain' header style to only contain page number; this is
% automatically applied to new \maketitle or \chapter pages
\fancypagestyle{plain}{%
  \renewcommand{\headrulewidth}{0pt}%
  \fancyhf{}%
  \fancyfoot[C]{\thepage}%
}

% customize header and footer contents
\fancyhead[L]{\fancyplain{} Rex Ying}
\fancyhead[R]{\fancyplain{} Differential Geometry}
\fancyfoot[C]{\thepage}

% words with accents that are annoying to type
\newcommand{\naive}{na\"{i}ve}
\newcommand{\visavis}{vis-\`a-vis}
\newcommand{\matlab}{\textsc{Matlab}}

% math operators
\DeclareMathOperator{\colspan}{colspan}
\DeclareMathOperator{\rank}{rank}
\DeclareMathOperator{\conv2}{\ast\ast}
\DeclareMathOperator{\corr2}{\star\star}

% language-specific hyphenation blocks
\newcommand{\en}[1]{\begin{english}#1\end{english}}
 
%--------------------------------------------------------------------
% Title
%--------------------------------------------------------------------

\title{\textbf{Digital Image Processing Project}}

\author{Rex Ying%
  \thanks{Undergraduate student, Department of Computer Science, Duke
  University, Durham, NC 27708.}\\
  \href{mailto:zy26@cs.duke.edu}{\texttt{zy26@cs.duke.edu}}
}

\date{\today}


%--------------------------------------------------------------------
% Document
%--------------------------------------------------------------------

\begin{document}

\maketitle
\onehalfspacing

% \label{sec:abstract}
% \phantomsection
% \addcontentsline{toc}{section}{Abstract}

Tentative project title: Use of wavelets in image processing on a group of similar images \\

The paper I found relates to texture detection and image segmentation based on wavelet-domain information.
The paper \emph{Image processing with complex wavelets} also talks about the limitations of using wavelets
transform, including shift dependence, as opposed to the property of Fourier transform. 
"Real DWTs are unlikely to
give consistent results when used to detect key features in images". 
The paper then presents the advantages of complex wavelets and how it could solve 2 main problems of DWT.
The other paper 
\emph{Multiscale Image Segmentation Using Wavelet-Domain Hidden Markov Models} has introduced a way to
use training data to solve this problem, among other uses. 
In lots of applications, the number of images is abundant, so I would like to explore if this can be used
to a significant advantage in terms of processing in wavelet domain.
The first paper envisions a range of applications
of the processing technique including motion estimation, denoising, texture analysis and synthesis, and object segmentation, but it only gives a single denoising example.
 I would like to try out segmentation in different context and assess its usefulness. \\

I also will have an internship with Google Maps this summer, so I will try to explore if this topic 
has relevance in terms of analysing and processing large amount of similar street views and satellite views. \\

Reference paper:

Kingsbury, Nick. "Image processing with complex wavelets." Philosophical Transactions of the Royal Society of London. Series A: Mathematical, Physical and Engineering Sciences 357, no. 1760 (1999): 2543-2560.

Hyeokho Choi; Baraniuk, R.G., "Multiscale image segmentation using wavelet-domain hidden Markov models," Image Processing, IEEE Transactions on , vol.10, no.9, pp.1309,1321, Sep 2001



%--------------------------------------------------------------------
% References
%--------------------------------------------------------------------

\bibliographystyle{plainnat}

\label{sec:references}
\phantomsection
\addcontentsline{toc}{section}{References}
\bibliography{@@Bib-file@@}

\end{document}



