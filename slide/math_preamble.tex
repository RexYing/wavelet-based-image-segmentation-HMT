%%%%%%%%%%%%%%%%%%%%%%%%%%%%%%%%%%%%%%%%%%%%%%%%%%%%%%%%
% 
% MATH ONLY
%
%%%%%%%%%%%%%%%%%%%%%%%%%%%%%%%%%%%%%%%%%%%%%%%%%%%%%%%%

% substitute
\usepackage{xstring}
\def\replaceStr#1{%
  \IfSubStr{#1}{,}{%
    \StrSubstitute{#1}{beingSubstituted}{desired}
  }{#1}
}

% Column vectors (any number of arguments)
% gives error in texclipse because of the pmatrix pair!
% \newcount\colveccount
% \newcommand*\colvecn[1]{
%         \global\colveccount#1
%         \begin{pmatrix}
%         \colvecnext
% }
% \def\colvecnext#1{
%         #1
%         \global\advance\colveccount-1
%         \ifnum\colveccount>0
%           \\
%           \expandafter\colvecnext
%         \else
%            \end{pmatrix}
%         \fi
% }
\usepackage{bm}
% ========= Vectors and matrices ===================
%
% Comment out to use arrow as vector notation
\renewcommand{\vec}[1]{\mathbf{#1}}
%\renewcommand{\vec}[1]{\bm{#1}}
%\renewcommand{\vec}[1]{\boldsymbol #1}
\newcommand{\colvec}[1]{\begin{pmatrix} #1 \end{pmatrix}}
% Matrix does not use bold now 
\newcommand{\mat}[1]{#1}
%\newcommand{\mat}[1]{\mathbf{#1}}
% transpose
\newcommand{\trsp}{^T}

% ========== Common sets ===========================
\newcommand{\real}{\mathbb{R}}
\newcommand{\complex}{\mathbb{C}}
\newcommand{\nat}{\mathbb{N}}
\newcommand{\integer}{\mathbb{Z}}

% derivative
\newcommand{\pder}[2]{\frac{\partial#1}{\partial#2}}
\newcommand{\pderop}[1]{\frac{\partial}{\partial#1}}
\newcommand{\der}[2]{\frac{\mathrm{d}#1}{\mathrm{d}#2}}
\newcommand{\derop}[1]{\frac{\mathrm{d}}{\mathrm{d}#1}}
\newcommand{\secpder}[3]{\frac{\partial^2 #1}{\partial #2 \partial #3}}
\newcommand{\npder}[3]{\frac{\partial^{#3} #1}{\partial #2^{#3} }}
\newcommand{\nder}[3]{\frac{\mathrm{d}^{#3} #1}{\mathrm{d} #2^{#3} }}

% equal by definition
\usepackage{mathtools}
\newcommand{\defeq}{\vcentcolon=}

% ============= Norm ==============================
%
% usage: \norm{ \biggl(\sum_{n=1}^N \mathbf{P}_{n}\biggr) }
\newcommand{\norm}[1]{\left\lVert#1\right\rVert}

% ===============Functions ========================
%
% Trigo
%
\newcommand{\sech}{\, \mathrm{sech} \,}
\newcommand{\sinc}{\, \mathrm{sinc} \,}


% math operators
\DeclareMathOperator{\colspan}{colspan}
\DeclareMathOperator{\rank}{rank}
\DeclareMathOperator{\conv2}{\ast\ast}
\DeclareMathOperator{\corr2}{\star\star}
% argmin/max
\DeclareMathOperator*{\argminOp}{argmin}
\DeclareMathOperator*{\argmaxOp}{argmax}
\newcommand*{\argmin}{\argminOp\limits}
\newcommand*{\argmax}{\argmaxOp\limits}

% =================== Fancy =======================
% Fourier transform operator
\DeclareMathOperator{\FT}{\mathfrak{F}}

